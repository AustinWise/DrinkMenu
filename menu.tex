\documentclass[12pt]{article}
\usepackage[utf8]{inputenc}
\usepackage[T1]{fontenc}
\usepackage[textwidth=12cm,centering]{geometry}
\usepackage[x11names]{xcolor}
\usepackage{background}

\newcommand*\wb[3]{%
  {\fontsize{#1}{#2}\usefont{U}{webo}{xl}{n}#3}}

% The page frame
\SetBgColor{Goldenrod3}
\SetBgAngle{0}
\SetBgScale{1}
\SetBgOpacity{1}
\SetBgContents{%
\begin{tikzpicture}
\node at (0.5\paperwidth,0) {\wb{80}{34}{E}\rule[60pt]{.2\textwidth}{0.4pt}%
  \raisebox{55pt}{%
  \makebox[.6\textwidth]{\ \fontsize{20}{29}\selectfont\scshape Drink Menu }}%
  \rule[60pt]{.2\textwidth}{0.4pt}\wb{80}{34}{F}};
\node at (2,-0.5\textheight) {\rule{0.4pt}{.8\textheight}};
\node at (19.5,-0.5\textheight) {\rule{0.4pt}{.8\textheight}};
\node at (0.5\paperwidth,-\textheight) {\wb{80}{34}{G}\rule[-10pt]{\textwidth}{0.4pt}\wb{80}{34}{H}} ;
\end{tikzpicture}%
}

% colorize text
\newcommand*\ColText[1]{\textcolor{Goldenrod3}{#1}}

% a tabular* for each food group
\newenvironment{Group}[1]
  {\noindent\begin{tabular*}{\textwidth}{@{}p{.8\linewidth}@{\extracolsep{\fill}}r@{}}
    {\fontsize{24}{29}\selectfont\ColText{#1}}\\[0.8em]}
  {\end{tabular*}}

% to format each entry
\newcommand*\Entry[1]{%
  \sffamily#1 & 5 \\
}

% to format each subentry
\newcommand*\Expl[1]{
  \hspace*{1em}\footnotesize #1 \\
}

\newcommand*\HowTo[1]{
%can comment this out for a prettyier menu
  HOWTO: \hspace*{1em}\footnotesize #1 \\
}

\pagestyle{empty}

\begin{document}



\begin{Group}{Snow Much Fun 2014}

\Entry{Green Circle}
\Expl{Also know as the japanese slipper.}

\Entry{Blue Square}
\Expl{A White Lady made with Blue Curaçao.}

\Entry{Black Diamond}
\Expl{TBD}

\end{Group}

\vfill

\begin{Group}{Winter Drinks}
\Entry{Hot Buttered Rum}

\Entry{Hot Toddy}
\HowTo{ whisky, boiling water and sugar or honey.}

\Entry{Irish Coffee}
\HowTo{2 parts Irish whiskey, 4 parts hot coffee, 1.5 parts fresh cream, 1 tsp brown sugar}

\Entry{Tom and Jerry}
\end{Group}

\vfill

\begin{Group}{New Drinks!}

\Entry{The End of the Road}
\Expl{The strong flavors in this drink work together better than the ingredident list would suggest. Laphroaig, Compair, Green Chartreuse}
\HowTo{Equal Parts.}

\Entry{The Last Word}
\Expl{I don't know if this is good but we are trying it. Gin, Lime Juice, Green Chartreuse, Maraschino Liqueur.}
\HowTo{Equal Parts. Bonus round: sub Laphroaig for gin.}

\end{Group}

\vfill

\begin{Group}{Austin's Favorites}

\Entry{Penicillin}
\Expl{Scotch, Ginger Liquor, Lemon, Honey, Laphroaig float}

\Entry{Chantily Lace}
\Expl{Plymouth Gin, Fresh Lime Juice, Small Hands Pineapple Gum, Apricot Brandy, Peychauds Bitters}
\HowTo{2 Plymouth Gin, 1 Fresh Lime Juice, 0.5 Small Hands Pineapple Gum, 1 Apricot Brandy, Peychauds Bitters}

\Entry{Shanghai Buck}
\Expl{Rum, Lime Juice, Giner Syrup, Ginger Ale}
\HowTo{2x rum, 1x lime, 1x ginger stuff, float ginger ale.}

\Entry{Solerno Corpse Reviver}
\Expl{A twist on the Corpse Reviver \#2. Gin, Lemon Juice, Solerno Blood Orange Liqueur, Lillet, dash absinthe}
\HowTo{Equal parts}

\end{Group}

\vfill

\begin{Group}{Bog Standard Drinks}
\emph{Standard drinks that can be made with the ingredients on hand.}

\Entry{The Mission Bell}
\Expl{A passion fruit Daiquiri.}
\HowTo{9x white rum, 5x lime, 3x passion fruit syrup}

\Entry{Daiquiri}
\Entry{Gin \& Tonic}
\Expl{The best drink on the menu TBQH.}
\Entry{Martini}
\Entry{Cuba Libre}
\Entry{Old Fashioned}
\Entry{Manhattan}

\Entry{Aviation}
\HowTo{4.5 cl Gin, 1.5 cl Lemon Juice, 1.5 cl Maraschino Liqueur}

\end{Group}


\end{document}
