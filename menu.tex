\documentclass[12pt]{article}
\usepackage[utf8]{inputenc}
\usepackage[T1]{fontenc}
\usepackage[textwidth=12cm,centering]{geometry}
\usepackage[x11names]{xcolor}
\usepackage{background}

\newcommand*\wb[3]{%
  {\fontsize{#1}{#2}\usefont{U}{webo}{xl}{n}#3}}

%color
%https://color.adobe.com/Cherry-Cheesecake-color-theme-2354/
\definecolor{MyColor}{rgb}{0.7254901960784314,0.0705882352941176,0.1058823529411765}
\definecolor{DarkColor}{rgb}{0.2980392156862745,0.1058823529411765,0.1058823529411765}

% The page frame
\SetBgColor{MyColor}
\SetBgAngle{0}
\SetBgScale{1}
\SetBgOpacity{1}
\SetBgContents{%
\begin{tikzpicture}
\node at (0.5\paperwidth,0) {\wb{80}{34}{E}
  \rule[60pt]{.1\textwidth}{0.4pt}
  \raisebox{55pt}{%
  \makebox[.8\textwidth]{\ \fontsize{24}{29}\selectfont\scshape Labor Day Drinks}}
  \rule[60pt]{.1\textwidth}{0.4pt}
  \wb{80}{34}{F}};
\node at (2,-0.5\textheight) {\rule{0.4pt}{.8\textheight}};
\node at (19.5,-0.5\textheight) {\rule{0.4pt}{.8\textheight}};
\node at (0.5\paperwidth,-\textheight) {\wb{80}{34}{G}
  \rule[-10pt]{.4\textwidth}{0.4pt}
  \raisebox{-12pt}{
    \makebox[.2\textwidth]{2016}}
  \rule[-10pt]{.4\textwidth}{0.4pt}
\wb{80}{34}{H}

} ;
\end{tikzpicture}%
}

% colorize text
\newcommand*\ColText[1]{\textcolor{DarkColor}{#1}}

% a tabular* for each food group
\newenvironment{Group}[1]
  {\noindent\begin{tabular*}{\textwidth}{@{}p{.8\linewidth}@{\extracolsep{\fill}}r@{}}
    {\fontsize{24}{29}\selectfont\ColText{#1}}\\[0.8em]}
  {\end{tabular*}}

% to format each entry
\newcommand*\Entry[1]{%
  \sffamily#1 \\
}

% to format each subentry
\newcommand*\Expl[1]{
  \hspace*{1em}\footnotesize #1 \\
}

\newcommand*\HowTo[1]{
% can comment this out for a prettier menu
  \hspace*{1em}\footnotesize How to: \hspace*{1em}#1 \\
}

\pagestyle{empty}

\begin{document}

\begin{Group}{New Drink!}
\Entry{Sloe Moon Raising}
\Expl{Nice berry flavors. Gin, sloe gin, framboise, lime, bitters.}
\HowTo{1 gin, 1 sloe gin, 1 framboise, 1 lime, dash bitters}
\end{Group}
\vfill

%\begin{Group}{From the Slopes}

\Entry{Green Circle}
\Expl{Also known as the japanese slipper.}
\HowTo{Equal parts Midori, Cointreau, and lemon juice}

\Entry{Blue Square}
\Expl{A White Lady made with Blue Curaçao.}
\HowTo{4cl gin, 3cl triple sec, 2cl lemon juice}

\Entry{Black Diamond}
\Expl{A glass of fernet.}
\HowTo{Fernet}

\end{Group}

%\vfill

%\begin{Group}{Classic Winter Drinks}
\Entry{Hot Buttered Rum}

\Entry{Hot Toddy}
\HowTo{Whisky, boiling water, lemon juice, and honey.}

\Entry{Irish Coffee}
\HowTo{2 parts Irish whiskey, 4 parts hot coffee, 1.5 parts fresh cream, 1 tsp brown sugar}

%\Entry{Tom and Jerry}
\end{Group}

%\vfill

\begin{Group}{Austin's Favorites}

\Entry{Penicillin}
\Expl{Scotch, Ginger Liquor, Lemon, Honey, Laphroaig float}
\HowTo{Honey, 1 whiskey, 0.25 ginger, 0.75 lemon}

\Entry{Chantily Lace}
\Expl{Plymouth Gin, Fresh Lime Juice, Small Hands Pineapple Gum, Apricot Brandy, Peychauds Bitters}
\HowTo{2 Plymouth Gin, 1 Fresh Lime Juice, 0.5 Small Hands Pineapple Gum, 1 Apricot Brandy, Peychauds Bitters}

\Entry{Shanghai Buck}
\Expl{Rum, Lime Juice, Ginger Syrup, Ginger Ale}
\HowTo{2x rum, 1x lime, 1x ginger stuff, float ginger ale.}

%\Entry{Corpse Reviver \#2}
%\Expl{Gin, Lemon Juice, Cointreau, Lillet, dash absinthe.}
%\HowTo{Equal parts}

\end{Group}


%\begin{Group}{Something Different}

% Inspired by https://sockpuppet.org/about/ , "The Final Word" which subs rye whiskey for the gin.
\Entry{The Last Word}
\Expl{Prohibition-era cocktail originally developed at the Detroit Athletic Club. Gin, Lime Juice, Green Chartreuse, Maraschino Liqueur.}
\HowTo{Equal Parts. Bonus round: sub Laphroaig for gin.}

\Entry{Aviation}
\Expl{The Aviation was created by Hugo Ensslin, head bartender at the Hotel Wallick in New York, in the early twentieth century}
\HowTo{4.5 cl Gin, 1.5 cl Lemon Juice, 1.5 cl Maraschino Liqueur}

% From Single Barrel in San Jose.
\Entry{The End of the Road}
\Expl{The strong flavors in this drink work together better than the ingredident list would suggest. Laphroaig, Campari, Green Chartreuse}
\HowTo{Equal Parts.}

\end{Group}

%\vfill

\begin{Group}{Standard Drinks}
\emph{Well known drinks that can be made with the ingredients on hand.}

%\Entry{The Mission Bell}
%\Expl{A passion fruit Daiquiri.}
%\HowTo{9x white rum, 5x lime, 3x passion fruit syrup}

%\Entry{Daiquiri}
\Entry{Gin \& Tonic}
\Expl{The best drink on the menu TBQH.}
%\Entry{Martini}
\Entry{Cuba Libre}
%\Entry{Old Fashioned}
%\Entry{Manhattan}

\end{Group}



\end{document}
